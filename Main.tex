\documentclass{article}
\usepackage{hyperref}
\usepackage{graphicx}
\hypersetup{
    colorlinks,
    citecolor=black,
    filecolor=black,
    linkcolor=black,
    urlcolor=black
}
\begin{document}
\title{NavUp Architectural Design}
\maketitle
		\begin{center}
			\textbf{\newline Team Java} \\
		\end{center}
			
				
		\begin{flushright} \large
		\end{flushright}
\clearpage
\tableofcontents
	
\clearpage
\section{Specific Requirements}
	\subsection{External Interface Requirements}
		\subsubsection{User Interface}
The UI(user interface) of NavUp will be designed to be user friendly. This is required to make the mobile application easy to use for the user. This will in turn reduce the time it takes the user to get accustomed to the mobile application and it's functionality. The UI will offer the base functionality to both a guest and logged-in user. A user will be able to easily log in or register to gain access to further functionality of the system.

		\subsubsection{Hardware Interface}
The hardware required will have to be in the form of a smartphone/tablet. The system will be designed with the idea of it being mobile. The system will thus be developed on a mobile platform for it to be downloaded as an application to be used on a smartphone or tablet. This will allow the system to be used on the go.

		\subsubsection{Software Interface}
The mobile application itself will be designed to function on the Android and IOS mobile platforms. These platforms have touchscreen capabilities, allowing the user to use the system as any other application on their smartphone. The built in software of each platform allows the system to be compacked, users will be able to easily navigate and search various locations by using the on-board keyboard of the smartphone.

		\subsubsection{Communications Interface}
As stated in the SRS document, the system will be using GIS navigation. It will also make use of wifi-signal strength to determine a user's position depending on the various wifi hotspots on campus. A user will make use of these technologies in various ways, but they need to be connected to the system in some form to gain access to it, this includes mobile data and wifi.
	\subsection{Performance Requirements}
	
	\subsection{Design Constraints}
	
	\subsection{Software System Attributes}

\section{UML Diagrams}
	\subsection{Users}
	
	\subsection{Navigation}
	
	\subsection{Notification}
	
	\subsection{Points of Interest}

\section{Design Pattern}
	\subsection{Design Patterns for Sub-systems}
	
\end{document}